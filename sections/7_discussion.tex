\section{Discussion}

\subsection{Supporting Artistic Skill-building via Enhanced Agency \& Control}
\todo{Alice's thought: I think we should have a section about the immense opportunity we found in designing tools to help artists build skills rather than like replace them -- we can maybe cite some of the learning sciences lit Haiyi was talking about for the proposal?}
\jh{agreed!}

\begin{itemize}
    \item At individual levels: 2d \& 3d artists perceive artistic control differently
    \item involving community engagement \& feedback or environmental factors
\end{itemize}

==== copied over from proposal, if we want to include in discussion ====

Our next aim is to use \textbf{scaffolding} techniques inspired by the learning sciences~\citep{gibbons2002scaffolding}, where temporary support is provided to help a learner complete a task beyond their current unaided ability, but gradually removed to improve the capabilities, creativity, and independence of novice artists over time. Applying scaffolding is a major challenge since modern  generative AI is not transparent, with limited abilities to `unlearn' knowledge~\citep{bourtoule2021machine}, and easy hackable~\citep{wei2023jailbroken}. To tackle this, we will design our system as a set of modules strictly partitioned to enable unlearning, and with strict guardrails so users cannot access removed capabilities during scaffolding. We will base our design on best practices in designing educational tools. Novice artists will have access to full scaffolding (e.g., step-by-step sketching, adaptations, and recombinations) at the beginning. As they progress, the scaffolding will be gradually removed. We will co-design and pilot-test the system with both novice and experienced artists, as well as educators. %\textcolor{red}{[TODO talk major impact human-centered challenges] [TODO What does artist upskilling mean and how to measure it]}. 

We will measure upskilling through a combination of self-assessment scales from the education literature, performance-based tasks that track improvement pre-and post- system integration as well as a series of longitudinal user studies. In addition to improving novice artist capabilities, our system will also improve their creativity through challenging them to deviate from historical styles and help creativity and personalization. More broadly, this creativity scaffolding has implications for other domains where human–AI collaboration is critical, including design, education, and knowledge work, pointing toward new ways of using AI not just to automate but to cultivate human creativity and growth.


% \subsection{agency/control - tools should help artist build skills!}




\subsection{Policy Recommendations for ethical development/training AI/compensation}


\subsection{Limitations \& Future Work}