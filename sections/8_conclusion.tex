\section{Conclusion}
This project set out to understand how novice visual artists begin their creative workflows, what challenges they face in the early stage, and where emerging generative AI could offer support. Through formative interviews and co-design workshops, we observed that novices already rely on an ecosystem of tools for gathering references, sketching first ideas, and testing directions. Building on these practices, we distilled a set of empirically grounded capabilities that artists want from generative AI to scaffold exploration, spark variety, and help them learn without taking away creative control. At the same time, the study surfaced tensions that shape whether and how generative AI belongs in the early stage. Participants voiced concerns about authorship, credibility, and the risk that AI may limit personal style. They also worried about the opacity of training data and the absence of reliable credit and compensation for the creators whose work may inform model outputs. These findings suggest that design and policy must advance together. Systems should make provenance and attribution legible, allow artists to set boundaries on source use, and create space for human-guided iteration. Policy should address data governance and artist compensation mechanisms so that early-stage assistance does not come at the expense of artistic labor and trust. Our results point to promising opportunities for tools that support brainstorming, reference curation, and skill-building while preserving agency and voice. Future work may prototype and evaluate tools with capabilities for reference gathering, guided variation, and explanation of process. Continued research may also study their effects on learning and confidence over time. In sum, we find opportunities for generative AI to play a constructive role in early-stage creative workflows if it is designed for partnership rather than replacement and grounded in policies that respect artists and their contributions.