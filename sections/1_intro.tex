\section{Introduction}
Visual art creation is a complex, multistage process requiring significant time, effort and material investment. 
Recent advances and uses of generative AI show potential for supporting this intensive creation process through consistent, guided generation and editing \cite{zhang2023text} of increasingly realistic and detailed visual outputs \cite{podell2024sdxl}. Such progress fuels excitement about potentials for visual generative models to revolutionize and democratize art (education) communities \cite{paradox, democratize, edu}. Despite the promise, recent scholarship studying the influence of generative systems on artists call for further engagement and foregrounding of input and opinions from the artistic community itself \cite{foreground, decolonial}.

The vast majority of visual generative AI tools in HCI scholarship explored ways to support creative workflows of designers \todo{cite a lot of studies}. But while designers create to serve commercial and professional purposes, non-professionals visual artists such as novices often create to develop new techniques and capabilities \cite{clinical, personal}, practice self-expression \cite{real_benefits} or challenge conventional aesthetic and creative boundaries \cite{synesthesia, art_implications}. The general-purpose and zero-shot nature of recent models makes artmaking accessible to audiences with little to no training, experience or knowledge \cite{lgtm} -- 
allowing them to produce high-quality and novel visual outputs 
at reduced costs and effort.
% But artmaking also benefits its creators -- helping them develop abstract capabilities (e.g., critical observation \cite{clinical} and reflection \cite{real_benefits}, creativity \cite{personal}, communication \cite{}, emotional expression \cite{}), and helping to enhance more measurable abilities such as academic performance \cite{grades} and memory \cite{memory}. 
% (as opposed to self-expression and -improvement in artistic techniques) 
\todo{motivation about upskilling novice workers.. reaching a broader community than only professionals.. potential for education etc. Hong: is it also fair to say that existing HCI work has largely focused on professional designers, while novice artists demand different form of support? Jane: yes we have evidence for this in 2.1}
% Such developments lead scholars to consider how generative systems may challenge conventional aesthetic and creative boundaries \cite{synesthesia, art_implications} 
% -- allowing novices \todo{we havent defined or motivated novices at this point yet} to produce high-quality and novel visual outputs without significant investment into materials or formal training.


Yet, existing generative AI systems often fall short in supporting novice visual artists with effectively integrating capabilities into their creative processes \cite{unstraighten}, leaving them stressed and disoriented \cite{impact, lgtm} -- despite works demonstrating its potential to augment early stages of artistic creation \cite{art_implications, poet, lamuse}. 
Few studies investigated how artists (especially novices \cite{review}) adopted co-creation and inspiration tools to meaningfully organize and reflect on inputs \cite{lamuse}, leaving a critical gap in understanding how recent advances can support novice visual artists in early-stages of their personal, expressive and educational practices -- which include processes such as immersion, reflection and research to spark inspiration \cite{botella2018stages}.
% \begin{itemize}
    % \item define early stage (from Botella et al. this is immersion, reflection, and research process)
    % \item current co-creation and inspiration tools remain limited in how they help visual artists meaningfully organize and reflect on inputs
    % \item existing tools don't yet align with artists' own workflows or artistic intentions during these early stages
    % \item 
% \end{itemize}
% \todo{merge with paragraph above on yet, existing generative } 
Meanwhile, artists remain deeply distrusting of generative systems due to potential downstream challenges influencing members of the artistic community, including economic and reputational harms -- raising questions about what artist-centered uses of generative AI tools could look like. 
\todo{Hong: maybe here we reemphasize the gap again: lack of understanding/support of novice workflows, limited understanding of how genai might support early-stage ideation, instead of final output; misalignemnt/distruct between artists and existing tools?}
% \todo{we dont talk about ethical challenges and harms..} as well as economic and reputational harms to their community members, raising urging questions about what accessible and equitable uses of generative AI tools could look like for artmaking. 

% \textcolor{red}{Describe here our first contribution: A descriptive account of early-stage workflows for novice visual artists}

% The literature, press and our own interview participants revealed how artist communities strongly resisted AI use – predominantly due to lack of attribution and accreditation \cite{credit} \todo{we dont talk about this also} (plagiarism \& copyright infringement) to the underlying artworks that are used to train and develop the model, as well as unpaid compensation to artists \cite{foreground}, resulting in financial and economic losses. Beyond legal and commercial concerns, environmental impacts and preferences for experience-inspired artworks also emerged as ethical and aesthetic concerns around AI-generated art \cite{versus}. 

To better align visual AI generation tools with artist workflows and intentions during these early stages, this study considers alternative and non-professional use cases
% Thus, instead of professional and commercial use cases such as design work, this study considers alternative uses of visual AI generation tools 
that directly serve the artist communities that they benefit and profit from. While prior work focused on AI art or final-image generation, we employed a two-step approach to explore ways of 
% generative AI tools might scaffold the 
% maybe incorporate into related works?
% Critical AI scholars point to the colonial \cite{decolonial}, imperial \cite{imperial} and anthropomorphic nature of commercial generative AI systems.
% Generative AI has the potential to lower barriers to engage in visual art and expand access to creative tools, yet early-stage support for novice visual arts remains under-suporte and offen inaccessible. 
% \todo{cite Ko et al.}
% \todo{adapt from 1-pager}
% Past empirical work with visual artists [9] found potential for LGTMs (large-scale text-to-image-generation models) to support artists in automating their creative processes, expanding idea exploration and mediating visual communication
% However, artists interviewed in this study found current LGTMs disorienting and hard to use
% Generative models are also found to lower barriers for visual storytelling
% [there are more works discussing democratization/access]
% [there are also many emerging techniques that may support casual/early-career artists]
% Through a two-step approach, we explored ways of 
integrating generative AI tools into the early-stage workflows of novice visual artists. First, we interviewed 15 novice artists about their existing practices to understand initial perceptions of available generative tools, as well as painpoints to their artistic creation processes. Then, we mapped these expressed needs to state-of-the-art technical advances, and presented four \todo{4 or 6?} functional prototypes using existing methods to probe for reactions and design objectives that artists prioritized for future forms of tooling support. These prototypes introduce state-of-the-art capabilities for medium and style transfer, visual combinations and editing, step-by-step tutorials and extensions of 2D into interactive 3D viewing.
% By bringing these capabilities together we significantly advance the state-of-the-art in 
% }

% In this current moment of resistance towards AI use in the artist community, it’s crucial to ponder: 
% what alternative worlds of practices — e.g., in training models and accrediting their sources – would artists require of model developers?
% What are acceptable use cases of such visual generative models?

% Activities/questions we might ask during codesign to approach this ^^
% preferred conditions/regulations/guardrails for generative tools, e.g.,
% Required disclosures of training sets used to produce outputs
% making sure outputs (and prompts) don’t belong to model creators
% [maybe this is a better question for professional artists, but as a thought question, what would an artist’s AI bill of rights look like?]

% \begin{itemize}
    % \item we combine interview and co-design methods to surface both pain points and desirable futures for novice-facing tools
    % \item online prior work focused on AI art or final-image generation, we explore how generative AI tools might scaffold \todo{add design objectives}
% \end{itemize}

Research questions:
\begin{enumerate}
\item[\textbf{RQ 1a}] 
\textbf{What processes do novice visual artists undertake to navigate early stages of their creative workflows, and what challenges do they encounter?}
\begin{enumerate}
    \item[\textbf{1b}] \textbf{Which of these early-stage challenges might align with emerging generative AI capabilities?}
    % are the needs for tool support that have in the early stages?
\end{enumerate}
% \item[\textbf{RQ 2}] 
% \textbf{What are the needs for tool support that novice artists have in the early stages?}
\item[\textbf{RQ 2}] 
\textbf{How do novice artists perceive opportunities for emerging capabilities to support their early-stage artmaking? What are considerations \dots \todo{iterate on this more once we have more workshop findings}
% What are the opportunities for supporting novice artists with generative AI capabilities in the early stages?
}
\end{enumerate}
Our contributions
\begin{itemize}
    \item Empirical illustration of tensions of how novice artists use tools in the early stage
    \item Design opportunities for creative support with generative AI tools for early-stage 
    \item prototypes here
\end{itemize}