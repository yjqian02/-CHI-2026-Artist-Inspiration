\section{Introduction}
Visual art creation is a complex, multistage process requiring significant time, effort, and material investment. Recent advances and uses of generative AI show potential for supporting this intensive creation process through consistent, guided generation and editing \cite{zhang2023text} of increasingly realistic and detailed visual outputs \cite{podell2024sdxl}. 
Such progress, coupled with lowered barriers of access, sparks excitement around the potential for visual generative models to revolutionize and democratize art education and expand creative participation \cite{paradox, democratize, edu}. 

While many creativity support tools enabled by visual generative AI have explored ways to augment the workflows of professional artists and designers across various domains \cite{wang2025aideation, GenQuery, architectural}, novice visual artists often engage in art-making to explore new capabilities \cite{clinical, personal}, practice self-expression \cite{real_benefits} or challenge conventional aesthetic and creative boundaries \cite{decline_novelty, art_implications}. For those users, the general-purpose and zero-shot nature of recent models offers an exciting promise: to make art-making accessible to audiences with limited training \cite{lgtm}. In theory, such tools could empower more people to develop skills, enjoy the process of art making, and gain a deeper appreciation for visual craft \cite{jackson}. 

Yet, existing generative AI systems often fall short in supporting novice visual artists, particularly in the early stage of the creative process. Prior work has highlighted the potential of AI to augment early ideation \cite{art_implications, poet, lamuse}, but novice artists often find these tools not being effectively integrated into their creative processes \cite{unstraighten}, leaving them stressed and disoriented \cite{kawakami2024impact, lgtm}. 
%Few studies investigated how artists (especially novices \cite{hu2025designing}) adopted co-creation and inspiration tools to meaningfully organize and reflect on inputs \cite{lamuse}, leaving a critical gap in understanding how recent advances may align with early-stages workflows of novice visual artists \cite{botella2018stages}.
% \begin{itemize}
    % \item define early stage (from Botella et al. this is immersion, reflection, and research process)
    % \item current co-creation and inspiration tools remain limited in how they help visual artists meaningfully organize and reflect on inputs
    % \item existing tools don't yet align with artists' own workflows or artistic intentions during these early stages
    % \item 
% \end{itemize}
% \todo{merge with paragraph above on yet, existing generative } 
Meanwhile, artists remain deeply distrusting of generative systems due to harmful downstream impacts on members of the artistic community (e.g., economic and reputational harms arising from inadequate attribution or compensation), raising questions about what artist-centered, ethically sound uses and developments of generative AI tools could look like -- especially for those who just beginning their creative journeys. 

To date, much of the HCI and UIST literature has focused on understanding and supporting final-stage outputs
of \textit{professional} designers and artists \cite{design_professionals, shi2023understanding}, while generative AI support for the early stages of novice visual artists workflows remain underexplored -- overlooking a key opportunity to bridge the gap between creativity support tools and emerging members of the artistic community. In particular, few studies have examined how novice artists \cite{hu2025designing} adopted co-creation and inspiration tools to meaningfully organize and reflect on inputs \cite{lamuse}, leaving a critical gap in understanding how emerging generative capabilities may align with novice visual artists' \textbf{early-stage workflows} involving practices that start at the definition of a piece and ending with the use of references for revisions to pieces  \cite{botella2018stages}.

To address these gaps, this study investigates how generative AI can meaningfully support the early-stage workflows of novice visual artists. 
% While prior work focused on AI-generated art \cite{explorers, machine} or final-image generation \cite{lgtm}, 
We employed a two-step approach: (1) We interviewed 15 novice artists to understand their existing practices, initial perceptions of available generative tools, as well as painpoints to their artistic creation processes (\S\ref{interview}); next (2) We mapped these expressed needs to state-of-the-art technical capabilities and develop six functional prototypes, which we evaluated in co-design workshops with 14 novice artists (\S\ref{codesign}).
Leveraging state-of-the-art capabilities (e.g., medium and style transfer, visual combinations and editing, step-by-step tutorials, and extensions of 2D into interactive 3D viewing),
these interactive prototypes allowed us to probe novice artists for reactions and design objectives they prioritized for future forms of tooling support. 


We ask the following research questions:
\begin{enumerate}
\item[\textbf{RQ 1}] 
\textbf{What processes do novice visual artists undertake to navigate early stages of their creative workflows, and what challenges do they encounter?}
% \item[\textbf{RQ 1b}] \textbf{Which of these early-stage challenges might align with emerging generative AI capabilities?}
\item[\textbf{RQ 2}] 
\textbf{How do novice visual artists perceive the potential of emerging generative AI capabilities to support their early-stage needs, and what tensions do they foresee around their integration?}
% \item[\textbf{RQ 2b}] 
% \textbf{What design tensions emerge when considering future advances and use cases of such capabilities?}

\end{enumerate}

% Building on these questions, we map early stage practices, probe emerging capabilities through co design and prototypes, and distill implications for tool, contributing an empirical account of how novice artists use tools in the early stage and the challenges they face, empirically grounded capability requirements for early stage support that artists want in tools, and a synthesis of the tensions these tools raise with implications for design and policy.

Building on the above questions, we map early-stage practice, probe emerging capabilities through co-design and prototypes, and distill implications for toolmakers. Accordingly, this paper contributes.
\begin{itemize}
     \item An empirical account of how novice artists use tools in the early stage and the challenges they face.
    \item Empirically grounded capability requirements for early stage support that artists want in tools.
    \item A synthesis of the tensions these tools raise and the implications for design and policy.
\end{itemize}