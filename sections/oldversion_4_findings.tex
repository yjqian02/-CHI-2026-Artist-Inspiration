

% ----------- dumping sections currently planning on cutting, but keeping just for now in case there are quotes we need ------------------------

% \subsubsection{Idea Exploration and Sketching}
% % Many participants engaged with sketching to explore their emerging ideas for a piece, yet participants expressed a desire to be able to more efficiently implement certain techniques to avoid having to redo certain parts of their pieces. Also, participants didn't always seem to have an established process for sketching, with our participants who had several years of experience in a specific type of art having an established process while others vaguely engaged in sketching and weren't aware of tool support or were intimated by the digital tools available. 

% For several participants, sketching served as a core method for exploring ideas, but the exact process participants followed in this phase varied widely, as some lacked clear structure in their workflow. Participants' sketching behaviors ranged from loose exploratory sketching to detailed implementations of certain elements (e.g., P5, P7, and P9). Participants with more experience in mediums that involved sketching had more defined sketching process, such as P9 who had been using the same process of pencil sketching following with inking for 5-6 years (P9). Participants expressed that there were certain unique challenges in this phase, such as difficulty redoing certain parts of a piece without completely starting over or managing tools: \textit{``I would have liked something that could have helped me catch that this arm had kind of [from a] wonky perspective before it got so far that I already colored and shaded. It would have saved me a lot of time.''} (P3). Ultimately, we found that sketching and idea exploration were a generative phase, but also a site of friction where participant skill, confidence, and literacy in tools eventually shaped how exploration unfolded. 


% \todo{paul comment: why do we start with this section/analysis?}


% \todo{paul - are these actually common? i thought many artists do not care about voice and gesture and AR etc.}

% \textbf{\textit{Once control is defined, describe the channels - sketch, voice, etc. through which control is exercised.}
% }The way artists interact with creative tools fundamentally shapes their experience, influencing not just efficiency but also the depth and quality of their engagement. Throughout the early stages of artistic motivation, artists express a desire for interaction modalities that are flexible and closely aligned with their natural workflows. Artists envision technology that seamlessly integrates into their creative environment, improving instead of disrupting their process. For example, some participants imagine using augmented reality (AR) glasses to overlay references or compositional guides directly onto their physical art pieces (P7) \todo{quote}. This modality would enable artists to visualize, manipulate, and combine digital references in real space, bridging the gap between the digital and physical mediums. Integration like this supports more hands-on approach to ideation and composition, allowing artists to experiment with new directions without breaking their workflows. Moreover, artists appreciate the potential of tools that accept speech or audio input that frees their hands to continue drawing or painting while searching for references, adjusting settings, or requesting feedback (P7) \todo{quote}. This hands-free approach not only increases efficiency but also feels more natural and less intrusive. Similarly, some artists (P6) express a preference for AI systems that accept experiential inputs, such as voice or sketch, over traditional text-based prompts, making the interaction feel more like a creative conversation than a technical command \todo{quote}. For the sketching stage that is often the most critical phase where foundational decisions about shape, proportion, and composition are made, artists value modalities that support quick and intuitive feedback either through real-time voice commentary, sketch overlays, or mixed-reality visualizations \todo{quote}.

% \textbf{\textit{relevant quotes or analysis from figma:}
% }consideration: imagining augmented reality glasses overlaying an art piece as a modality of interaction - P7
% consideration: tools where you can interact via speech and audio are nice because you're hands can be free to do art then - P7
% consideration: the sketching stage of a piece is most important since it lays the foundation, this is where feedback about shapes/proportions/etc is helpful - P7
% wants inputs to AI be more experiential than text -- e.g., voice-based - P6



% \subsubsection{Control and Creative Agency}
% \textit{\textbf{Starts first since retaining authorship is the thread that ties all other stages together.}}
% Throughout every stage of the artistic process, maintaining control and creative agency is a foundational need for artists. Rather than surrendering authorship or creative input to digital tools, artists seek systems that act as supportive enablers, helping them achieve their goals while preserving ownership and the freedom to make personal decisions. The desire for agency is evident not only in how artists approach inspiration and ideation, but also in how they interact with automation, references, feedback, and learning resources. For example, when gathering references, they want granular control over search filters and the ability to select or manipulate specific images, sometimes envisioning tools as flexible as blender, where references can be “puppeted” or stylized to suit their vision. \todo{quote} During visualization and composition, manual authority over what is changed, preserved, or combined is crucial, especially when merging multiple sources. As P13 articulated, manual control is especially valued in combining reference images to ensure that the final direction is always artist-led. \todo{quote} Artists prefer to determine when and how automated features assist their process, with the option to step back and take over at any moment. P14 finds AI-generated or incomplete art inspiring and wants tools that provide partially finished references as long as they retain control over the final composition and details. \todo{quote}Neutral or low-quality references are sometimes preferred precisely because they leave room for imagination and personal style, rather than establishing a predefined standard for aesthetic. Artists also value the ability to accept or reject suggestions, and to decide the pace and scope of tutorials according to their own needs and interests. P11 finds tools the most helpful when they act as teachers or guides not replacements, \todo{quote} automation is acceptable only when it improves education and growth without diminishing creative control or authorship

% \textbf{\textit{relevant analysis from figma below:}}
% P14 finds AI or incomplete art amusing and inspiring, and would like tools that show partially finished versions of art to help guide their process, but still wants to complete the final details themselves; P14 is cautious about tools that provide too much detail, want to maintain creative control over the initial composition
% P14 would like AI tools that provide incomplete references, so they can maintain control over the final style and composition

% P14 would find AI process tools useful if they provide quick/neutral references that don’t influence their own style or ideas; they prefer references that are unopinionated and focus on composition rather than style, even if the images are low quality, because this helps them maintain creative control
% P13 values manual control over her creative process, especially when combining reference images
% need: way to have specific reference images and to control them in something like blender (i.e., puppet but stylized the way you want it to be) - P8
% P11 wants tools to act as teachers or guides, not as replacements; and she’s open to some automation if it’s educational, but wants to maintain creative control and authorship over her work
% P14 is cautious about using AI in their creative process and values maintaining personal control over their art
% consideration: support for artists should be akin to spell correct on word document where it does not stifle the artist's original voice - P8


% \section{\todo{4 Findings New Structure}}
\textbf{}



\section{Findings}

In the following section, we illustrate findings from the first phase of our study. We report on the specific practices that novice visual artists currently engage in during the early stages of their creative workflows. Next, we highlight key opportunities for generative AI support in these stages to support the creativity of novice artists. 
%  TODO: findings summary
% RQ1: What specific practices do novice visual artists currently engage in during the early stages of their creative workflows?
% RQ2: What are opportunities (interview) and challenges (co-design) in supporting the creativity of novices in the early-stage visual art process? (Interviews + co-design)

\subsection{RQ1: What specific practices do novice visual artists currently engage in during the early stages of their creative workflows?}

%  why do artists create? 
% internal drive, sustained commitment to making art 
% immersion stage from Botella et al.
\subsubsection{Motivations for Engaging in Art-Making}
Participants were primarily motivated by the internal satisfaction of creating, describing art-making as a personally meaningful, \textbf{calming, and reflective} activity. Participants described frequently turning to visual art creation even without a concrete outcome in mind, using the process as a way to relax or process emotions. For instance, P14 describes how ``\textit{a lot of the art that I create on a slightly more frequent basis \dots is for journaling purposes}'', while P4 also perceived artmaking as
``\textit{a very solo experience \dots the thing I do alone to connect with myself and make something that makes me feel better about the world \dots when I'm not getting enough alone time}''. 

Beyond serving reflective purposes, P10 also describe the calming benefits of regularly practicing a familiar concept: ``\textit{I really enjoy drawing old toys. I don't know why. It just a relaxing thing for me, all the puppets, with some broken pieces and so on. And it's the thing that's more comfortable for me to draw somehow}''.  Similarly, P14 offers an example of a longheld ``\textit{small practice called window that I've kept around, on and off since, maybe 2017 and it's super simple: I have a small rectangle or a small square, and I know that I can fill that square with anything I want that represents that particular day}''.

For some, motivation was shaped by \textbf{moments of recognition or affirmation }from others. Several participants credited teachers, friends, or family members with encouraging them to create visual art. For instance, P10 recalled taking a class where the teacher
``\textit{ask[ed] me there}, `oh my gosh, are you an artist? This looks fantastic!' \dots \textit{So I started looking at Tiktok videos of it, and I was able to start painting with it}''

Additionally, a few participants described feeling compelled to create visual art after \textbf{encountering art by other artists}. This form of motivation sparked a desire in participants to engage in their own creative process. As P12 explained, seeing an inspiring piece of work \textit{``ma[de] me think of something that I've never thought of before. It's usually something that I like and I feel inspired by, and [then] I feel creative.''} Together, these findings highlight that motivation to make visual art among novice artists emerges from a blend of intrinsic enjoyment, relational encouragement, and a desire to respond creatively to the world around them.


% what gets artists started on a specific piece
% external stimulus or trigger, spark of idea or visual concept
% in the research stage in Botella et al.
\subsubsection{Gathering Inspiration and References}
% while gathering inspiration from a mix of real-life and digital resources such as google search and pinterest was part of several of our participants' workflows, they often experienced frustration with limited functionalities of search mechanisms
Participants frequently gathered inspiration from both real-life and digital resources such as Google Images and Pinterest, yet some found existing tools frustratingly limited in helping them to surface relevant or high-quality reference materials. Several participants explained that searching for reference images on Google Image Search (P1, 5, 12), Pinterest (P1, 2, 5, 9, 11, 
13), and social media (P5, 7, 9, 11, 12, 13
) as part of their regular workflow. When gathering these references, participants were sometimes seeking broader goals such as stimulation for ideas to create a piece:
\textit{``there's making a concept which needs to be understandable, accessible, and visually simulating at the end of it''} (P6). For other participants, references served as important technical models for poses of objectives or specifically human bodies and expressions (P4, P5, P14). 

%  here since this is the understanding of practices section, we should focus on the practices (the workarounds participants used like photoshopping or just pasting multiple reference images into their workflow)
Across participants who engaged in the digital search process, a common pattern of frustration emerged around the limitations of current search mechanisms. Participants employed workaround methods to obtain the types of references they wanted in these cases. One of the key workaround strategies was to iteratively edit search terms on Google Search or Pinterest. Several participants utilized this strategy to manually refine results to filter for specific types of references they wanted, such as those in a specific medium (P4), photos that feature realistic human emotions (P8), and photos taken from a certain camera angle (P5) or featuring a certain pose a person is striking (P4 and P5). A few participants searched for a specific reference image they were picturing, such as P2 who searched for a specific type of blueberry she was picturing (P1, P2, and P5). 

Additionally, many participants made manual changes to reference images. For example, several participants combined reference images to form one cohesive reference image. For example, a few participants manually combined images that made up different objects in their planned art piece through photoshopping a combination of images (P5 and P13). P9 combined references from real-life photos with references with a certain artistic style they wanted to follow (P12). Our participants also employed strategies to edit and interact with a single reference image. For example, P7 developed an app to allow them to blur certain details in reference photos. Additionally, P14 would make their own reference photos, such as by taking photos of their hands when they could not find a reference that was from the perspective and pose they were imagining.  
Lastly, some participants described having to scroll through all of the reference images they saved, whether they were from search engine results or photos they took themselves. In these cases, some participants liked the experience of revisiting all their reference images (P9), but at the same time, manually browsing their reference library was time-consuming (P6 and P9).   

\todo{do we want to add a paragraph about the need for quality control (i.e., what's AI what isn't?)}


\subsubsection{Idea Exploration and Sketching}
% Many participants engaged with sketching to explore their emerging ideas for a piece, yet participants expressed a desire to be able to more efficiently implement certain techniques to avoid having to redo certain parts of their pieces. Also, participants didn't always seem to have an established process for sketching, with our participants who had several years of experience in a specific type of art having an established process while others vaguely engaged in sketching and weren't aware of tool support or were intimated by the digital tools available. 

For several participants, sketching served as a core method for exploring ideas, but the exact process participants followed in this phase varied widely, as some lacked clear structure in their workflow. Participants' sketching behaviors ranged from loose exploratory sketching to detailed implementations of certain elements (e.g., P5, P7, and P9). Participants with more experience in mediums that involved sketching had more defined sketching process, such as P9 who had been using the same process of pencil sketching following with inking for 5-6 years (P9). Participants expressed that there were certain unique challenges in this phase, such as difficulty redoing certain parts of a piece without completely starting over or managing tools: \textit{``I would have liked something that could have helped me catch that this arm had kind of [from a] wonky perspective before it got so far that I already colored and shaded. It would have saved me a lot of time.''} (P3). Ultimately, we found that sketching and idea exploration were a generative phase, but also a site of friction where participant skill, confidence, and literacy in tools eventually shaped how exploration unfolded. 

\subsubsection{Skillbuilding and Learning from Others}
% Participants often drew directly from the work of other artists to improve their skills, yet depending on their level of experience with a certain medium or style, often struggled to implement skills on their own without support
Our participants often relied on mimicking or adapting others' work to build skills, but some found it challenging to translate observation into implementation. Participants explained established practices such as `artist studies' in which they would recreate a piece by an artist whose style and techniques they hope to learn, or following tutorials, or adapting pieces with different styles. For some participants, translating observation of a piece was challenging, particularly if the piece was in a medium they were not familiar with or if they were attempting to translate the piece from a one style to another. For instance, P2 said that \textit{``I'm not an art professional, so sometimes I would still struggle with techniques of painting certain things ... originally, I was learning traditional Chinese watercolor, and that's very different from Western watercolor''} (P2). 

\subsubsection{Digital and AI Tool Usage}
% digital tool usage varied, with some participants who were used digital tools for a long time being experienced with it while others expressed a desire to use them more but were intimated by them. Participants expressed reservations for using AI tools [cite literature] resonating with prior literature about issues with how AI models were trained without adequate credibility to artists or compensation as well as broader concerns such as that for the environment. However, several participants were also hopeful of the potential of tools to be integrated in the creative support process in specifically the early stages. Interesting, some participants used "AI-based tools" e.g., procreate's gradient coloring tool but did not consider that "AI" 

Participants expressed a mix of excitement and hesitation toward digital and AI-supported tools: while some users embraced advanced digital and AI features, others felt intimated by unfamiliar digital tools, and most held complex attitudes toward the ethical implications of generative AI tools. 
Several of our participants were well-versed with digital tools such as Procreate, Adobe Photoshop, and more. Most of these participants were self-taught, with familiarity with these tools shaping their confidence in using them. For some participants, even non-AI based digital tools felt inaccessible and intimating: \textit{``I'm not good with computers, so I don't really mess with [digital tools]''} (P9).  Moreover, ethical concerns around artist credit, artist compensation, and the environmental cost of AI usage, reflecting concerns surfaced in prior research and popular media, were echoed by our participants. For some participants, the ethical issues around how AI tools made it difficult for them to imagine themselves ever using AI tools. Case in point, P4 struggled to get around the environment issues of LLMs: 
\begin{quote}
    \textit{``Every AI image that's generated is really environmentally destructive, and I think that sucks. There's no way around that. There's no way to make this system better that prevents the environmental destruction of every single generated crop. And I struggle with that.''} (P4)
\end{quote}
Other participants expressed an openness to AI tools if certain conditions were met. P8 expressed that they were only open to using AI tools if an ethical compensation model was implemented: \textit{``If there was a tool that was created by artists for artists, where ... [there are] royalties, like a book, where the artist approves and can control [use of] their art ... I would definitely use it.''} (P8)
Broadly, we surface a key tension in a desire for support in better tools for visual artists along with a need to retain control, authorship, and concern for ethical issues. 
 % In this way, some participants were curious and even hopeful about AI integration in the early-stage processes of visual art creation. \todo{(add ideas participants suggested like pose suggestions, auto shadowing, etc.)}

\subsection{RQ2: What are opportunities and challenges in supporting the creativity of novices in the early-stage visual art process?}

%  maybe for each of these paragraphs we can talk about the opportunity then the challenges; should we reverse to talk about challenges first then opportunities to introduce future tool implications as part of the opportunities?
\subsubsection{Quicker Access to Better References}
A persistent challenge for novice artists is the difficulty of finding reference images that truly align with their creative intent. Participants expressed a desire for systems that support searching for references in a way that more closely matches how they imagine their art. For example, [quote] wished for the ability to input a rough sketch or combine a simialr image with a short text description to surface references that fit their mental model, rather than relying solely on keyword-based searches.

Participants also highlighted the need for more natural and intuitive ways to gather references. [quote] Rather than sifting through generic search results, [quote] sought systems that could help them organize and incorporate real-life references, such as images tied to meaningful personal experiences or emotional moments. This includes the ability to filter references by specific attributes, such as focusing on human faces or bodies [quote], finding images relevant to a particular digital tool [quote], or surfacing perspectives or poses that are otherwise hard to locate [quote].

Moreover, participants frequently described the challenge of noise in current reference search tools. [quote] They wanted to filter results by visual similarity, such as silhouette, style, or medium, or to quickly exclude irrelevant or distracting images. [quote] By allowing users to specify and refine the kinds of references they need, future tools could better support the early stages of creative exploration, helping novice artists move from inspiration to ideation with more confidence and convenience.

P1: "So firstly, when I'm when I didn't have a really good idea. I only, so firstly, what? I only have a very rough idea about story of the girl and well, I will search on the on Pinterest, see, oh, with, like a, I mean, maybe the girl has killed the Well, I would search, oh, how do people kill each other on Pinterest? So that's the forming of the idea."
P14: "imagining for the particular drawing of the woman falling, for example, it would have been really useful to have a pose of that thing falling in kind of a realistic style, although, as I'm thinking about that, as you know, almost as if it looks like an image of that kind of puppet, that posable puppet, reaching into the air with that particular pose from the angle that I wanted to be as well being able to specify"
P12: "I usually just Google the objects I want to paint and pick the image that’s closest to what I imagined… otherwise I don’t even know how many petals that flower should have."
P8: "having that in, like, in 3d life, but with, with, like, blender, blender, or it's like a 3d or like modeling thing, yeah, so, but like, instead being able to move it like a puppet, I think that would be great to be able to, like, say, like, be mad, or like, move the face to make it mad, and then you can sketch that out. I think that would be great."
P13: "So this was originally where I took inspiration from. I thought that was like a such a cool piece I was gonna sell this. I was gonna sell my art at the sunflower field event that they had. So I just drew up some sunflowers with this kind of like, similar style. I just thought I just look at the photo and just really see like, what, what do I like about it, basically, and what, what do I need to do to make it look like this style? And so if I look at it right, you can see that, basically, there is line, line art, right? Right? And it's really kind of thick in some areas. It's usually pretty thick sometimes, but then you could see, like, kind of like the creases, they gotta be thin. And then I thought was really cool about it was the, kind of, like, the how they use contour lines right to show the the shape or the form of the leaf. And so I see that contour line curves here. I thought that was really cool. And so that's what I incorporated into my sunflower piece."
P11: "If I’m drawing a character in an interesting pose I usually just go to Pinterest… the character-design boards there are always really good."
P12: "I don’t actively look for inspiration; the algorithm just suggests things to me on Instagram, and that’s what I save for later."

\subsubsection{Perspective from Another Artist}
P11: "the software is kind of like a teacher or like a critiquer that corrects you as you do different types of things."
P12: "moment that I’ve I think I often feel very creative in good discussions, like when I’m really feel like The other pace person is resonating with my like discussions, in the sense of like discussion about new ideas. So like, when we’re trying to ideate, and when I feel another person is really resonating with me, not necessarily with my idea, but just generally with me. And like, where I want to push this project to, that’s where I feel very creative."
P10: "But since I somehow left that word, I completely stopped to receive that kind of feedback and support, because I’m I didn’t have the chance to connect with people that are involved in that part. So I’m not surrounded by artists, actually. So I don’t have basically anyone to ask for support. I mean, I can always ask for advice or comments from my friends, of course, but I don’t think that they are really objective somehow. Yeah, okay, I always show them what I mean. I used to show them what I draw and so on. But I know that maybe they don’t have the right background to really suggest me particular modification, and that they suggest you just come from their own opinion, which are not from unprofessional or so on."
P12: "I do. I do. I think I asked friends: Oh, do you think this perspective is off? What something is not right about this image? Can you help me to figure out what’s wrong with it? Because some things are just technical, like [with] this, the proportions are just off. The eye is in the wrong position, for example, and I just can’t see it. But then when I’m trying to get a specific style, I try to analyze the original painting as much as I can. For example, when I’m trying todo something in the style of Raphael, a Renaissance painter, I would probably take a really, really close look at their painting and try to look at that image, because I’m obviously not in Florence. I can’t look at the real thing. ButI would take a very close look at their painting and try to recreate their technique"

\subsubsection{Visual Combination of References}

P12: "usually the images that I look up are not paintings, but photographs. So then translating the photograph into a drawing is, I guess, a process you call painting. Or you can call painting translating the real thing into a style is what you do while you paint. Usually if I’m trying to paint in the style of a person, I have their painting and then the real object that I wanted to paint next to each other, and then I try to kind of understand how would referral approach, like a palm tree, for example"


\subsubsection{Support to Learn Techniques and Styles}