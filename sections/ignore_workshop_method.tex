\subsection{Phase 2 Codesign Methods} % Cindy• Individual sessions that validated Taxonomy v1 (all needs) in one diagram 

After developing the taxonomy of needs (\S ref needs table), mapping them to emergent generative capabilities, and creating corresponding probes (\S\ref{probes}), we introduced the (1) taxonomy of needs and (2) six probes to participants via an onboarding session. At the beginning of sessions, we provided the taxonomy of four needs, along with\todo{@Jane are you still working on this? - from Alice haha}
\jh{sorry i crashed accidentally TT}

% • Literature-informed-add-ons of plausible capabilities/considerations added from prior work before workshops
% • Meanwhile, participants completed HW to try out 6 lightweight prototypes as provocations we asked them to do before workshop

\subsubsection{Pre-workshop Prototype Interaction}
Before participating in co-design workshop sessions, participants were invited to 30-min onboarding sessions where we presented the 6 probes as prototypes and asked to reflect on questions. The purpose of this additional step was to allow participants an opportunity to utilize prototypes on their own time, as they would use tools within their natural visual art workflows. To ensure participant comfort, we advised participants that although we would not explicitly collect data from their art pieces, we could not guarantee the underlying models used in our prototypes would not do so \tood{maybe move this somewhere else}. \todo{add the questions we had in the slides}. We also viewed observations from participants prior to workshop sessions to better inform how we would facilitate the discussions. 

\subsubsection{Co-Design Workshops with Novice Artists}
To further investigate the tensions that might arise when implementing tools to address the four challenges identified in our study, we conducted four workshops with 3–4 participants each. Participants were recruited first from our interview pool, and later through social media postings and snowball sampling via referrals. The workshops were designed to create space for artists to converse with one another, compare experiences, and collectively reflect on the promises and complications of emerging AI tools. Each session began with participants reflecting on six prototype concepts\todo{add explanation of co-design methodology and cite examples of other papers that have done this}. We asked them to discuss challenges that might emerge when using such tools in their own practices, as well as opportunities for improvement. After this exercise, we introduced participants to examples of state-of-the-art tools identified through a survey of recent HCI and creativity support research\todo{need to explain how we did this selection}. These examples were not intended to be comprehensive, but rather to highlight representative directions in current research and to prompt further reflection on potential frictions \todo{this is alice's attempt at getting ahead of reviewers, others please help revise}. Finally, throughout the discussions, we encouraged participants to consider the broader ethical and practical implications of generative AI systems in art-making. In particular, we asked them to reflect on risks that may arise for novices and communities, as well as to share any recommendations for how such systems could better support creative growth while mitigating unintended consequences.



% no commercially available tools for needs 4.2-4.4, but they did have access to google search, etc.
