\section{Phase 1 Findings} % alice


\subsection{Novice Visual Artists' Early-Stage Practices and Broader Contexts}
We report below specific practices that novice visual artists currently engage in during the early stages of their creative workflows. \todo{maybe add a summary here and also some statement about how the stages of art-making our participants engaged in resonate with prior lit}


\subsubsection{Motivations \& Communities for Art-Making} \paul{i think this subsection could be shortened}
Compared to professional or outcome-driven goals, novice participants were more motivated by intrinsic enjoyment, affirmation from others, and encounters to art that \textit{spark} inspiration. Several shared internal satisfaction of creating as motivations, describing art-making as a personally meaningful, \textbf{calming, and reflective} activity; others frequently turned to visual art creation as a way to relax or process emotions, without a concrete outcome in mind. For instance, P14 describes how \textit{``a lot of the art that I create on a slightly more frequent basis \dots is for journaling purposes''}, while P4 perceived art-making as
\textit{``a very solo experience \dots the thing I do alone to connect with myself and make something that makes me feel better about the world \dots when I'm not getting enough alone time''}. 
Beyond reflections, P10 relates the calming benefits of regularly practicing a familiar concept: ``\textit{I really enjoy drawing old toys. I don't know why. It['s] just a relaxing thing for me, all the puppets, with some broken pieces  \dots it's the thing that's more comfortable for me to draw somehow''}.  
% As such, we found that an intrinsic value in art creation, such as the effect of being able to connect with themselves, motivated several of our participants to create visual art. 

\jh{there's some literature on environmental factors impacting artmaking that we can reference here or in discussion}
For some, motivation was shaped by \textbf{moments of external recognition or affirmation}. Several novices credited teachers, friends, or family members who encouraged them to create art. P10 recalled a class where the teacher
\textit{``ask[ed] me there}, `oh my gosh, are you an artist? This looks fantastic!' \dots \textit{So I started looking at Tiktok videos of it, and I was able to start painting with it''} (P10). Several felt compelled to create visual art after \textbf{encountering other artists' pieces}, such encounters made novices like P12 to\textit{``think of something that I've never thought of before. It's usually something that I like and I feel inspired by, and [then] I feel creative''} and motivated to make their own art. Together, these findings highlight how novice artists' motivation emerges from a blend of intrinsic enjoyment, relational encouragement, and a desire to respond creatively to the world around.

% \subsubsection{Art Contexts and Established Communities}
Participants also related desires for \textbf{broader connection to other artists and communities}, as well as existing practices of gathering input from others in their lives -- e.g., friends and family.
% was a practice several of our participants described either participating in or wanting to do more of. 
% Several participants described their 
P12 \textit{``asked friends sometimes `do you think this perspective is off? Like, what's something [that] is not right about this image? Can you help me figure out what's wrong with it? \dots [Because if] the eye is in the wrong position, for example, I just can't see it''} (P12). For others, being around other artists and their work was a beneficial source of inspiration: 
\begin{quote}
    \textit{``the thing that's authentically inspiring more than anything is actually just seeing other art. I feel extremely inspired to create when I see other artists[' work], whether it's photography, sculpture, or painting. I feel very inspired when I see other people create stuff, and it makes me want to create stuff.''} -P4
\end{quote}

In one interesting case, P11 enjoyed drawing with \textit{Art Fight}\footnote{} \paul{missing footnote?} --- a platform bringing artists online together to create different versions of each other's work, which P11 describes as \textit{``a drawing event that happens in July. There's teams. Everybody joins a team, and you draw other people's characters, and you get points for it \dots it's the game''}. 
Other important sites of connection to artist (communities) included formative exposure to art education (P5 and P9), online platforms (P2, P7), and informal peer networks (P4, P8). For example, P5 explained how \textit{``I get inspiration from other artists. Right now, I'm on Instagram a lot, so when I browse my export page and different reels, I'm like: `}wow, those people are so talented, and they do it so well\textit{'. And then I get inspired''} (P5). 
Many novice artists interviewees found value in sharing art, seeing other artists' work, and discussing their work with others --- experiences that built their confidence in artistic abilities and contributed senses of belonging in broader creative communities: ``\textit{my friend has a more a lot of experience in what is realistic for artists to do and is more experienced than me, and so has fallen into a lot of the same traps and knows what to watch out for}'' - P3.

% not sure if we want to include this quote
\paul{this paragraph is strong and directly motivates some of the ai parts so maybe can be brought earlier/above can be shortened} But while some described strong reliance on peer and community support, others reflected on the absence of such networks, leading to desires to partake in more robust or ongoing art communities.
% --spaces to learn collaboratively, exchange ideas, and grow alongside others. 
P4 expressed how \textit{``I would love the opportunity to make art with other more serious artists, where there was some more trust built between us''} (P4). 
This blend of community engagement and collaborative learning highlights nuanced ways novices sought connection and validation, underscoring complex needs for connection, including desire for (1) inspiration from other artists' works, (2) engaging with them to build relationships and provide support. 


\subsubsection{Digital/AI Tool Usage \& Ethical Considerations}
% Participants’ creative practices were deeply influenced by their 
Uneven access to digital tools (and self-taught approaches to learning them) deeply influenced novices' creative practices -- leaving them intimidated or under-equipped. 
Toward digital and AI-supported tools, participants expressed a mix of (1) \textbf{excitement} to embrace advanced digital and AI features, (2) \textbf{intimidation} by unfamiliar tools, and (3) most held \textbf{complex attitudes toward the ethical implications} of generative AI tools. 
% Several participants were self-taught and well-versed with digital tools (e.g., Procreate, Adobe Photoshop).
% and their familiarity shaping their confidence in using them. 
For some, even non-AI-based digital tools felt inaccessible and intimidating: \textit{``I'm not good with computers, so I don't really mess with [digital tools]''} (P9). Others expressed an openness to AI tools under certain conditions --- P8 required an ethical compensation model, describing \textit{``a tool that was \textbf{created by artists for artists}, where \dots [there are] royalties, \dots where the artist approves and can control their art \dots I would definitely use it.''} 
% Through these findings, we surface that our participants held a range of familiarity and willingness to engage in digital or state-of-the art tools and technologies for the early stages of visual art creation. 
 % In this way, some participants were curious and even hopeful about AI integration in the early-stage processes of visual art creation. \todo{(add ideas participants suggested like pose suggestions, auto shadowing, etc.)}


% \subsubsection{Ethical Considerations for AI Usage}
On the bright side, some recognize how AI tools offer genuine efficiency and convenience.
For tasks like searching large reference databases, AI can help reduce the overwhelm of manual browsing (P6).
For professional artists, tools can make reference creation easier or automate technical aspects of image creation (P13). 
% or automating technical aspects of image creation.  and
P8 is open to using AI in contexts where ethical compensation and transparency are prioritized, and P6 saw its value for handling large amounts of visual data. 
% In certain cases, AI-generated images or suggestions can help artists experiment with new directions, automate repetitive tasks, or serve as neutral references that don’t impose a specific style.
However, many artists remain wary of the impact on creative vision, professional identity, and integrity of the artistic community. 
% One recurring concern is the influence of generated images or limitations to an artist’s own creative vision. 
P14 reflected on how generating a character they created changed their perception of it:
\begin{quote}
    \textit{`` This [generated output] was a collage of generated things from Midjourney \dots This is completely inaccurate to how I first imagined her. She had a much rounder face. She looked kind of like a really sh*tty doll with completely overdone blush and eye shadow like absurd amounts very long fake eyelashes \dots Even though I still have this mental image that has by now faded to some extent \dots and I'm sad that I didn't articulate that myself first before being biased by [the generated image]''} - P14
\end{quote}
The degree of polish of AI images also deviate from artists' personal intentions or intuitions, threatening their sense of authorship and spontaneous freedom. 
P13 and P11 expressed frustration at how \textbf{uniform, polished, or illogical} generated content tended to be for learning or inspiration, which can be misleading due to anatomical or compositional errors. 
P12 and P4 consider emotional impact and authenticity of art to be foundational to artist’s story and process, which cannot be replicated by AI-generated work. 
% Artists want to connect with the inspirations, techniques, and narratives behind human-made art for sense of community and belongingness. 
P9 and P6 share how AI-generated images rarely inspire them the way that real-world experiences or handmade works do, whereas the pleasure of searching for references and abstracting from reality cannot be replaced by AI generation.

Some novices expressed discomfort around environmental costs and and the lack of permission or compensation, making it difficult for them to imagine themselves ever using AI tools. P4 shares her struggle: 
\begin{quote}
    \textit{``Every AI image that's generated is really environmentally destructive, and I think that sucks. There's no way around that. There's no way to make this system better that prevents the environmental destruction of every single generated crop. And I struggle with that.''} (P4)
\end{quote}
P8 and P11 were particularly concerned about the \textbf{lack of ethical compensation models} and the \textbf{use of copyrighted material}, with P11 emphasizing the importance of authenticity and integrity in the artist community --- AI-generated art for profit is seen as undercutting traditional incomes for artists. Others expressed worries that AI is encouraging an increasingly uniform culture that devalues original works (P10 and P6). More solution-oriented participants wanted clear labeling to avoid accidentally using AI-generated references (P13). 
\begin{table}[t]
\centering


\begin{tabular}{|p{4cm}|p{7cm}|p{5cm}|}
\hline
\textbf{Challenges} & \textbf{Suggestions} & \textbf{Relevant Existing Tools} \\
\hline

\textcolor{purple}{\textbf{4.1}} Quicker and Better Access to References &
\begin{itemize}
  \setlength\itemsep{0pt}   % reduce space between bullets
  \setlength\parskip{0pt}
  \item [4.1.1] Search using sketches or combined text+image inputs
  \item [4.1.2] Filter results by attributes
  \item [4.1.3] Organize personal and real life references
\end{itemize} &
\begin{itemize}
  \setlength\itemsep{0pt}
  \setlength\parskip{0pt}
  \item Semantic search
  \item add citations
\end{itemize} \\
\hline

\textcolor{purple}{\textbf{4.2}} Visualizing Combinations of References &
\begin{itemize}
  \setlength\itemsep{0pt}
  \setlength\parskip{0pt}
  \item [4.2.1] Merge multiple images into one reference
  \item [4.2.2] Interactive editing such as angle and style
\end{itemize} &
\begin{itemize}
  \setlength\itemsep{0pt}
  \setlength\parskip{0pt}
  \item Semantic combination
  \item Transformation
  \item add citations
\end{itemize} \\
\hline

\textcolor{purple}{\textbf{4.3}} Feedback from Another Artist's Perspective &
\begin{itemize}
  \setlength\itemsep{0pt}
  \setlength\parskip{0pt}
  \item [4.3.1] Peer like suggestions on references or styles
  \item [4.3.2] Surface similar works with annotations
\end{itemize} &
\begin{itemize}
  \setlength\itemsep{0pt}
  \setlength\parskip{0pt}
  \item Personalized recommendations
  \item learning tools
\end{itemize} \\
\hline

\textcolor{purple}{\textbf{4.4}} Support to Learn Art Techniques and Styles &
\begin{itemize}
  \setlength\itemsep{0pt}
  \setlength\parskip{0pt}
  \item [4.4.1] Step by step breakdowns of styles
  \item [4.4.2] Progressive suggestions during drawing or painting
\end{itemize} &
\begin{itemize}
  \setlength\itemsep{0pt}
  \setlength\parskip{0pt}
  \item Semantic decomposition
  \item Progressive generation
\end{itemize} \\
\hline
\end{tabular}
\caption{Summary of Phase 1 Findings and connection to existing tools}
\label{tab:phase1-summary}
\end{table}



\subsection{Areas for Support in the Early-Stages} % validation findings need to go here 

% \textbf{need}
% \textbf{consideration}
% • Some scenario(s) of concrete moment(s) of artist current practice to bring out/surface N1
% • What they do now (brief description + quotes)
% • Opportunity or tool capabilities that align (brief description + quotes that drawn out what they have/what they need but don't have/aren't sure is possible)
% • Tool considerations (today + future) of current workarounds, future envisions, guardrails
% \textit{\textbf{Repeat for N1-N2-N3-N4 for all our needs}}

Below, we present key challenges participants expressed experiencing, organized into four categories \paul{state the 4 categories?}. For each challenge, we describe complex difficulties novice artists experienced and present examples of workarounds some used to adapt current tools or practices to fit their needs. 
Next, we follow-up with suggestions participants provided for ways to be supported in artmaking. 
We conclude by presenting considerations for the development of possible tools, including envisioned challenges and suggested methods for mitigating possible harms or misuses of tools. 

\subsubsection{Quicker and Better Access to References} \paul{first few sentences are a bit long-winded}
Participants frequently struggled to find reference images that match their creative intent, reporting current tools as frustratingly limited in surfacing relevant or high-quality material.
% context of current practices
Participants frequently gathered inspiration from real-life and digital resources (e.g., Google Images, Pinterest), yet some found existing tools frustratingly limited in surfacing relevant or high-quality reference materials. Several participants considered the practice of searching for references on Google Image Search (P1, 5, 12), Pinterest (P1, 2, 5, 9, 11, 
13), and social media (P5, 7, 9, 11, 12, 13) a part of their regular workflow. P13 reflected that as a novice artist \textit{``I have a hard time drawing from imagination, and so what I do is I go on Pinterest and look up pictures''} (P13). 
%When gathering these references, participants were sometimes seeking broader goals such as stimulation for ideas to create a piece: \textit{``there's making a concept which needs to be understandable, accessible, and visually simulating at the end of it''} (P6). 
For others, references served as important technical models for poses of objects, or specific human bodies and expressions (P4, P5, P14). 

% explain the challenge
The difficulty of finding reference images that aligned with creative intent posed a persistent challenge for several participants, and many desired systems that facilitate reference searching in ways that more closely match their visions. 
As P5 explained: \textit{``the ideal kind of [tool would be] if I could just imagine a picture and then like it came to life''} (P5). Others wanted more complex filters when finding references, indicating specific desires to filter for elements such as human poses or expressions (P5, P8). 
Participants also highlighted needs for more natural and intuitive ways to gather references. 
P8 described a system where they could visually input preferences for recommendations on references:
\begin{quote}
    \textit{``like an Instagram where you can change your preferences at any time. There's check boxes of what you like \dots [so you can] pick the images that you think you like best that you would want your art to look like \dots that's better than having words [to search with], because sometimes words mean different things for other people.''} - P8
\end{quote}

Another challenge participants faced was the need to filter noise in current reference search tools (4.1.2). For example, P7 explained how this lack of support can prevent them from creating art altogether:
\begin{quote}
    \textit{``It's hard to extract the most important parts from [a reference image]. That leads to two problems. [One] is that sometimes I just don't like [to] paint certain scenes because I think it's too busy. \dots It takes me a long time to choose the right thing to paint. ''} - P7
\end{quote}
As such, one challenge is for participants to be able to specify and refine the kinds of references they need. 

% strategies and proposed solutions
To manage these gaps, participants developed a range of workarounds, from adjusting their search strategies to creating custom references. In describing these practices, some participants brought up suggestions for how they imagined tools should ideally support them. 
Across participants who engaged in the digital search process, a common pattern of frustration emerged around the limitations of current search mechanisms. Participants employed workaround methods to obtain the types of references they wanted in these cases. One of the key workaround strategies was to iteratively edit search terms on Google Search or Pinterest. Several participants utilized this strategy to manually refine results to filter for specific types of references they wanted, such as those in a specific medium (P4), photos that feature realistic human emotions (P8), and photos taken from a certain camera angle (P5) or featuring a certain pose a person is striking (P4 and P5). A few participants searched for a specific reference image they were picturing, such as P2, who searched for a specific type of blueberry she was picturing. P8 explained their process of iteratively improving their search keywords: 
\begin{quote}
\textit{``I look up like [the word] `upset,' and then I get disappointed because the pictures are not what I want. So then I keep thinking about it, and I become, I get more and more precise, like `upset man staring out the window beings. And then usually, I really start thinking of movies or things I've seen at that point that kind of give me that vibe. And then I think about that, or about games I played and what vibe they have. When pictures [of real] faces [from Google search] let me down, I look up [works from] actual artists.''} - P8
\end{quote}

Additionally, many artists made manual changes to reference images. For example, a few manually combined images that made up different objects in their planned art piece through photoshop (P5 and P13). P9 combined references from real-life photos with references with a certain artistic style they wanted to follow (P12). Some also employed strategies to edit and interact with a single reference image. For example, P7 developed an app to allow them to blur certain details in reference photos, but expressed a desire for a more complex tool: \textit{``it would be really cool if a system could take in a picture and just get the main essence of the of the kind of like scene from there, and get rid of all the unnecessary details''} (P7).
Additionally, P14 would make their own reference photos, such as by taking photos of their hands when they could not find a reference that was from the perspective and pose they were imagining.  
Lastly, some artists described having to scroll through all of the reference images they saved, whether they were from search engine results or photos they took themselves. In these cases, some participants liked the experience of revisiting all their reference images (P9), but at the same time, manually browsing their reference library was time-consuming (P6 and P9).   


% \todo{paul - seems very long quotes do we need all of them?} 

\todo{Alice working on this, please don't look at sections 4.2.2 - to end of phase 1 findings until done}

\subsubsection{Visualizing Combinations of References}
Another challenge participants faced was having easy methods of combining multiple references into a coherent image, describing difficulties in translating photographs, styles, or poses into a unified artistic vision. Some participants described using tools such as Adobe Photoshop to combine reference images together. For example, P5 explained why they took time to combine references:
\begin{quote}
    \textit{``[I] was using different images and combining them all together. It takes time to Photoshop it all together. \dots I can just take those images separately and paint or draw them as draw them together, even though they're separate. [However] I think it would be easier to combine them but again, it's just too much work.''} - P5
\end{quote}
P9, who expressed earlier that they were not comfortable using digital art tools, iteratively incorporated reference images into their pieces (P9). These efforts to stitch together multiple static references underscored both the importance of having cohesive visual guides and the limits of existing tools, setting the stage for participants’ interest in more dynamic or interactive solutions.

Some participants used interactive references so as to have a single place to meet the need of what they would otherwise need multiple references to capture differences in lighting or other aspects of a reference they may want to vary. For example, P1 explained that if they needed a reference \textit{``for human anatomy, I would personally go to download an app, like a medical app \dots because it's a 3d map. You can just rotate around the body, all the muscle. \dots you can see all the muscles and how they work''} (P1). These examples illustrate how existing tools to help participants combine or interact with reference images often provided only fragments of what participants needed, sometimes leading them to use workarounds for how pieces might fit together. While such strategies helped participants approximate their vision, they often required significant effort to interpret or manually blend references.


% opportunities participants suggest
While participants like P1 knew how to where to go to get such references, other participants expressed a desire for a dedicated and accessible tool for artists. Rather, some participants envisioned interactive systems that would allow them to manipulate poses, perspectives, or stylistic features in real time. When reflecting on an art piece they completed, P14 expressed that \textit{``it would have been really useful to have a pose of that thing falling in kind of a realistic style''} suggesting that such a tool could be a sort of \textit{``puppet reaching into the air with that particular pose from the angle that I want''} (P14). P8 had a similar need and said they would like to be able to manipulate a human-like puppet and \textit{``move the face to make it mad''} (P8). 
These suggestions highlight participants’ desire for tools that not only provide static references but also allow them to manipulate and combine references dynamically.



% \todo{paul - this sections seems incomplete}


\subsubsection{Perspective and Feedback from Another Artist}
Participants struggled with identifying errors or areas for improvement on their own, and often lacked timely, supportive feedback that matched their artistic intentions. Participants expressed that there were certain unique challenges in this phase, such as difficulty redoing certain parts of a piece without completely starting over or managing tools: \textit{``I would have liked something that could have helped me catch that this arm had kind of [from a] wonky perspective before it got so far that I already colored and shaded. It would have saved me a lot of time.''} (P3). This reflects a broader challenge of how artists access feedback and whose perspectives they can rely on.

% suggestions from participants
Several participants consistently express a desire for feedback that is context-aware, flexible, and aligned with their intentions for a given piece. For many, feedback is most valuable when it comes from someone with technical expertise or an “artist’s eye”, someone who can offer objective critique on aspects like balance, composition, or technique, rather than just compliments. For example, P8 often sought input from her boyfriend, an art major, to benefit from his formal training, while P4 appreciated informal encouragement and more serious critique in group art settings. For these participants, there was immense value in outside perspectives, especially when they had become too focused on their own work and needed fresh eyes to spot issues or opportunities. For some participants the desire for feedback extended beyond a need for technical critique. Compliments, encouragement, and casual observations from peers played a meaningful role, especially in group settings or informal art gatherings. For example, P4 explained that sometimes disagreements with perspectives from others were helpful: \textit{`` you might not initially be in alignment with somebody. They might tell you something [about your art]. You're like, `okay, I could see that.' [You] try that and you [realize] `oh, actually, I love it''} (P4). Together, these accounts illustrate that feedback was not only about technical accuracy but also about motivation and perspective-taking, highlighting the multifaceted role of outside input in sustaining creative practice.


Participants imagined tools that could replicate the sensitivity of an ‘artist’s eye,’ offering feedback that was flexible, context-aware, and tailored to their intentions. For example, P11 wanted a tool that could help encourage and motivate them: \textit{``[When] I want to give up, the tool is like, `no, you can do it''} feeling that such a tool would \textit{``be so fun. It'd be so cute, too. I feel like people would like that''} (P11). \todo{can add more ideas for this from the validation sessions} Overall, in imagining these tools, participants emphasized the need for both constructive and uplifting feedback, surfacing opportunities for tools that provide the sharpness of an artist’s eye and also tools that support artists with reassurances.



\subsubsection{Support to Learn Art Techniques and Styles}
Participants also described struggles with learning new techniques and styles on their own, with some finding it difficult to bridge the gap between observing others’ work and applying those methods in practice. This challenge was especially acute when artists attempted to adapt across mediums or translate from one stylistic tradition to another. For instance, P2 noted that \textit{``I'm not an art professional, so sometimes I would still struggle with techniques of painting certain things ... originally, I was learning traditional Chinese watercolor, and that's very different from Western watercolor''} (P2). To manage these difficulties, participants relied on practices such as copying interpretations of specific subjects when first learning (P2), recreating works through “artist studies,” following tutorials, or adapting pieces into different styles. Watching time-lapse or process videos (P11, P12) also helped demystify the effort behind finished pieces, making the process feel more approachable. Yet, translating observation into implementation was not always straightforward. Some participants expressed frustration when tutorials or advice were either too advanced or not grounded in the practical realities of their tools and materials (P2, P7). As such, participants emphasized the importance of resources that can be tailored to their current goals and level of experience, whether through step-by-step guidance, adaptive feedback, or examples that reveal works-in-progress rather than only finished pieces.


To address this challenge, participants envisioned tools that could scaffold learning by breaking techniques into progressive steps, tailoring feedback to their level, and providing contextual guidance as they practiced. P7 described an ideal tool they would use:
\begin{quote}
    \textit{``ideally you have this tool that's watching over you while you paint, or while you are making art. It learns to provide you feedback at the right moment so that you're not being handheld the whole way, but, at the same time, you're not like getting lost. At the same time it figures out what [you're] good at, what you still need to spend some timeimproving on.''} - P7
\end{quote}
\todo{opportunity here to add more data from validation sessions}
Such opportunities highlight a desire for resources that are not only instructive but also responsive to an artist’s goals and intentions.


