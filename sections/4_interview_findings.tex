\section{Phase 1 Findings} % alice


\subsection{Novice Visual Artists' Early-Stage Practices and Broader Contexts}
We report below specific practices that novice visual artists currently engage in during the early stages of their creative workflows.
\subsubsection{Motivations \& Communities for Art-Making.}
Compared to professional or outcome-driven goals, novice participants were more motivated by (1) \textbf{intrinsic enjoyment} and (2) \textbf{interactions with others} that provided support and sources of inspiration. For instance, P6 described how \textit{``a lot of the art that I create on a slightly more frequent basis \dots is for journaling purposes''}, while P4 perceived art-making as \textit{``a very solo experience \dots  [that] I do alone to connect with myself and make something that makes me feel better about the world''} (P4). Beyond reflections, P10 related the calming benefits of art-making: \textit{``I really enjoy drawing old toys. \dots It['s] just a relaxing thing for me''} (P10). For others, motivation was shaped by moments of external recognition or affirmation, with several novices crediting teachers, friends, or family members who encouraged them to create art. P10 recalled a class where the teacher
\textit{``ask[ed] me there}, `oh my gosh, are you an artist? This looks fantastic!' \dots \textit{So I started looking at Tiktok videos of it, and I was able to start painting with it''}. Several took up visual art after seeing other artist's work. Such experiences made novice artists like P12 \textit{``think of something that I've never thought of before. \dots and [then] I feel creative''} (P12). Together, these findings highlight how novice artists' motivation emerges from a blend of intrinsic enjoyment, relational encouragement, and a desire to respond creatively to the world around. 

Participants also gather input from people in their lives and sought broader ties with artists and communities. P12 asked friends \textit{``what's something [that] is not right about this image? \dots [Because if] the eye is in the wrong position, for example, I just can't see it''}. For others, being around other artists and their work was a beneficial source of inspiration (e.g., P4). In one interesting case, P11 enjoyed drawing with \textit{Art Fight}\footnote{https://artfight.net/} --- a platform bringing artists online together to create different versions of each other's work, which P11 described as \textit{``a drawing event that happens in July. Everybody joins a team, and you draw other people's characters, and you get points for it \dots it's the game''}. 
Other important sites of connection included art education (P5 and P9), online platforms (P2, P7), and informal peer networks (P4, P8). 

Crucially, while some participants described strong reliance on peer and community support, others reflected on the absence of such networks, leading to desires to partake in more robust or ongoing art communities.
P4 expressed how \textit{``I would love the opportunity to make art with other more serious artists, where there was some more trust built between us''}. 
This blend of community engagement and collaborative learning highlights nuanced ways novices sought connection and validation, underscoring complex needs for connection, including a desire for inspiration from other artists' works, engaging with them to build relationships and provide support. 

% -- things alice cut below in most recent pass--
% For example, P5 explained how \textit{``I get inspiration from other artists. Right now, I'm on Instagram a lot, so when I browse my export page and different reels, I'm like: `}wow, those people are so talented, and they do it so well\textit{'. And then I get inspired''} (P5). 
% Many novice artists interviewees found value in sharing art, seeing other artists' work, and discussing their work with others --- experiences that built their confidence in artistic abilities and contributed senses of belonging in broader creative communities: \textit{``my friend has a more a lot of experience in what is realistic for artists to do and is more experienced than me, and so has fallen into a lot of the same traps and knows what to watch out for''} (P3). 
% \textit{``I feel extremely inspired to create when I see other artists[' work], whether it's photography, sculpture, or painting. I feel very inspired when I see other people create stuff, and it makes me want to create stuff''} (P4). 
% \jh{there's some literature on environmental factors impacting artmaking that we can reference here or in discussion}
% \aq{agreed! let's move to discussion since we are focusing on what's novel about our findings specifically about novice artists here}


\subsubsection{Digital/AI Tool Usage \& Ethical Considerations.}
% \aq{part of me wonders if we should choose quotes that better highlight the novice aspect here. E.g., some participants talk about how they don't do art for commission, but they hear other professional artists in the communities they are in complaining about this, so they, in turn, are concerned about ethics. But if folks think it's okay, we can leave it}
% \jh{if they bring out contrast of novice versus professional practice, then yes we should include because of the novelty}
% Participants’ creative practices were deeply influenced by their 
Uneven access to digital tools (and self-taught approaches to learning them) deeply influenced novices' creative practices -- leaving them intimidated or under-equipped. 
Toward digital and AI-supported tools, participants expressed a mix of (1) \textbf{intimidation} by unfamiliar tools and (2) \textbf{excitement} to embrace advanced digital and AI features but also (3) \textbf{nuanced views} on the ethics of generative AI tools.  
For some, even non-AI-based digital tools felt inaccessible and intimidating: \textit{``I'm not good with computers, so I don't really mess with [digital tools]''} (P9). Other participants expressed excitement about the practical upsides of AI, looking forward to uses such as AI reducing the burden of manual browsing to find inspiration and allowing faster and easier manipulation of references (P13). 


% One recurring concern is the influence of generated images or limitations to an artist’s own creative vision. 
% Others expressed an openness to AI tools under certain conditions For example, P8 required an ethical compensation model, describing \textit{``a tool that was \textbf{created by artists for artists}, where \dots [there are] royalties, \dots where the artist approves and can control their art \dots I would definitely use it.''} 
However, many remained wary of the impact of using generative AI tools on creative vision, professional identity, and the integrity of the artistic community. 
P6 described an instance of a generated image reshaping their mental vision of a character in unintended ways, where the output \textit{``is completely inaccurate to how I first imagined her. She had a much rounder face. \dots I still have this mental image [of the character]\dots I'm sad that I didn't articulate that myself first before being biased by [the generated image].''} Others added that generated content often appeared uniform, overly polished, or illogical for learning or inspiration, with anatomical and compositional errors that could mislead practice (P13, P11). Several worried that AI encourages cultural sameness that devalues original work (P10). Concerns about professional identity centered on authorship and livelihoods. Participants cited the lack of ethical compensation models and the use of copyrighted material, and emphasized authenticity as a core value of the artist role. Some viewed AI generated art sold for profit as undercutting traditional income streams (P8, P11). 

Finally, participants raised issues of community integrity and infrastructure. Some wanted clear labeling to avoid inadvertently using AI generated references in their process (P13), others expressed discomfort with environmental costs and systems that operate without permission or compensation, making it hard to imagine adopting these tools: 
\begin{quote}
    \textit{``Every AI image that's generated is really environmentally destructive and that sucks. There's no way around that. There's no way to make this system better that prevents the environmental destruction of every single generated crop. And I struggle with that.''} (P4)
\end{quote}
Overall, participants weighed uneven access and limited familiarity against excitement about practical capabilities, while holding nuanced ethical positions on authorship, compensation, environmental impact, and community integrity.


% P12 and P4 consider emotional impact and authenticity of art to be foundational to artist’s story and process, which cannot be replicated by AI-generated work. 
% Artists want to connect with the inspirations, techniques, and narratives behind human-made art for sense of community and belongingness. 
% P9 share how AI-generated images rarely inspire them the way that real-world experiences or handmade works do, whereas the pleasure of searching for references and abstracting from reality cannot be replaced by AI generation.

% \begin{table}[t]
% \centering
% \begin{tabular}{|p{4cm}|p{7cm}|p{5cm}|}
% \hline
% \textbf{Challenges} & \textbf{Suggestions} & \textbf{Relevant Existing Tools} \\
% \hline

% \textcolor{purple}{\textbf{4.1}} Quicker and Better Access to References &
% \begin{itemize}
%   \setlength\itemsep{0pt}   % reduce space between bullets
%   \setlength\parskip{0pt}
%   \item [4.1.1] Search using sketches or combined text+image inputs
%   \item [4.1.2] Filter results by attributes
%   \item [4.1.3] Organize personal and real life references
% \end{itemize} &
% \begin{itemize}
%   \setlength\itemsep{0pt}
%   \setlength\parskip{0pt}
%   \item Semantic search
%   \item add citations
% \end{itemize} \\
% \hline

% \textcolor{purple}{\textbf{4.2}} Visualizing Combinations of References &
% \begin{itemize}
%   \setlength\itemsep{0pt}
%   \setlength\parskip{0pt}
%   \item [4.2.1] Merge multiple images into one reference
%   \item [4.2.2] Interactive editing such as angle and style
% \end{itemize} &
% \begin{itemize}
%   \setlength\itemsep{0pt}
%   \setlength\parskip{0pt}
%   \item Semantic combination
%   \item Transformation
%   \item add citations
% \end{itemize} \\
% \hline

% \textcolor{purple}{\textbf{4.3}} Feedback from Another Artist's Perspective &
% \begin{itemize}
%   \setlength\itemsep{0pt}
%   \setlength\parskip{0pt}
%   \item [4.3.1] Peer like suggestions on references or styles
%   \item [4.3.2] Surface similar works with annotations
% \end{itemize} &
% \begin{itemize}
%   \setlength\itemsep{0pt}
%   \setlength\parskip{0pt}
%   \item Personalized recommendations
%   \item learning tools
% \end{itemize} \\
% \hline

% \textcolor{purple}{\textbf{4.4}} Support to Learn Art Techniques and Styles &
% \begin{itemize}
%   \setlength\itemsep{0pt}
%   \setlength\parskip{0pt}
%   \item [4.4.1] Step by step breakdowns of styles
%   \item [4.4.2] Progressive suggestions during drawing or painting
% \end{itemize} &
% \begin{itemize}
%   \setlength\itemsep{0pt}
%   \setlength\parskip{0pt}
%   \item Semantic decomposition
%   \item Progressive generation
% \end{itemize} \\
% \hline
% \end{tabular}
% \caption{Summary of Phase 1 Findings and connection to existing tools}
% \label{tab:phase1-summary}
% \end{table}



\textbf{\textit{Fig 1. Initial Taxonomy v1 after interviews (original needs/what participant mentioned/with lit examples of possible capabilities/considerations)}}

\subsection{Areas for Support in the Early-Stages} % validation findings need to go here 

% \textbf{need}
% \textbf{consideration}
% • Some scenario(s) of concrete moment(s) of artist current practice to bring out/surface N1
% • What they do now (brief description + quotes)
% • Opportunity or tool capabilities that align (brief description + quotes that drawn out what they have/what they need but don't have/aren't sure is possible)
% • Tool considerations (today + future) of current workarounds, future envisions, guardrails
% \textit{\textbf{Repeat for N1-N2-N3-N4 for all our needs}}

Below, we present four key challenges: (1) 
\textbf{quicker and better access to references}, (2) \textbf{visualizing combinations of references}, (3) \textbf{perspective and feedback from another artist}, and (4) \textbf{support to learn art techniques and styles}. For each, we describe complex difficulties novice artists experienced and present examples of workarounds some used to adapt current tools or practices to fit their needs. 
Next, we follow-up with suggestions participants provided for ways to be supported in artmaking. 
We conclude with considerations for potential tool development, including envisioned challenges and suggested methods for mitigating and preventing potential harms or misuses. 

\subsubsection{Quicker and Better Access to References.} Participants frequently struggled to find reference images that match their creative intent, reporting current tools as frustratingly limited in surfacing relevant or high-quality material.
% context of current practices
% Participants frequently gathered inspiration from real-life and digital resources (e.g., Google Images, Pinterest), yet some found existing tools frustratingly limited in surfacing relevant or high-quality reference materials. 
Several participants regularly search references on Google Image Search (P1, 5, 12), Pinterest (P1, 2, 5, 9, 11, 13), and social media (P5, 7, 9, 11, 12, 13). P13 reflected how, as a novice artist \textit{``I have a hard time drawing from imagination, and so \dots I go on Pinterest and look up pictures''} (P13). 
For others, references were important technical models for poses of objects, or specific human bodies and expressions (P4, P5, P6). 

% explain the challenge
The difficulty of finding references that aligned with creative intent posed a persistent challenge for participants, who highlighted needs for more natural and intuitive ways to gather references. P8 envisioned
% \begin{quote}
    \textit{``an Instagram where you can change your preferences \dots There's check boxes of what you like \dots [so you can] pick the images \dots that's better than having [to search with] words, because sometimes words mean different things for other people.''} (P8).
% \end{quote}
As P5 explained: \textit{``the ideal kind of [tool would be] if I could just imagine a picture and then like it came to life''} (P5). Others wanted \textbf{fine-grained search filters for specific elements} like human poses or expressions (P5, P8). 
%Beyond inclusions of references through recommendations and filters,
Artists also cared about \textit{filtering out} noise in current tools (4.1.2). For novices like P7, this lack of support caused enough cognitive load to prevent them from creating art altogether:
\begin{quote}
    \textit{``It's hard to extract the most important parts from [a reference image]. That leads to two problems. [One] is that sometimes I just don't like [to] paint certain scenes because I think it's too busy. \dots It takes me a long time to choose the right thing to paint. ''} - P7
\end{quote}
% As such, one challenge is for participants to be able to specify and refine the kinds of references they need. 

% strategies and proposed solutions
% To manage these gaps, participants developed a range of workarounds, from adjusting their search strategies to creating custom references. 
% In describing these practices, some participants brought up suggestions for how they imagined tools should ideally support them. 
% Across participants who engaged in the digital search process, a common pattern of frustration emerged around the limitations of current search mechanisms. 
To manage these gaps, participants developed various workarounds to obtain desired references. Strategies included iteratively editing search terms on Google Search or Pinterest to manually refine attributes --- e.g., specific mediums (P4), objects (P2), realistic human emotions (P8), particular camera angles (P5) certain poses (P4, P5). 
% A few participants searched for a specific reference image they were picturing, such as P2 a particular type of blueberry. 
% P8 explained their process of iteratively improving search terms: 
% \begin{quote}
% \textit{``I look up like [the word] `upset,' and then I get disappointed because the pictures are not what I want. So then I keep thinking about it, and I become, I get more and more precise, like `upset man staring out the window beings. And then usually, I really start thinking of movies or things I've seen at that point that kind of give me that vibe. And then I think about that, or about games I played and what vibe they have. When pictures [of real] faces [from Google search] let me down, I look up [works from] actual artists.''} - P8
% \end{quote}
Many artists also manually changed reference images, using Photoshop to combine different objects in their planned art (P5, P13); P9 combined real-life references with certain artistic styles (P12). Others directly edited and interacted with a single reference  --- P7 developed an app to blur certain details in reference photos, but expressed a desire for more complex tooling: \textit{``it would be really cool if a system could take in a picture and just \textbf{get the main essence of the of the scene, and get rid of all the unnecessary details}''} (P7).
%P6 makes their own reference photos by taking photos of their hands when they could not find a desired perspective or pose.
% reference that was from the perspective and pose they were imagining.  
% Lastly, some artists described having to scroll through all of their saved references, whether they were from search engine results or photos they took themselves. 
Lastly, participants described the manual experience of scrolling through their references (e.g., search engine results, own photos) -- while some enjoyed revisiting all their reference images (P9), others found manual browsing of their reference library to be time-consuming (P9).   


% \todo{paul - seems very long quotes do we need all of them?} 
\subsubsection{Visualizing Combinations of References.}
Another challenge participants faced was combining multiple references into a coherent image, describing difficulties with translating photographs, styles, or poses into a unified artistic vision. For example, P5 explained the challenges of combining reference images with existing tools like Adobe Photoshop:
\begin{quote}
    \textit{``[I] was using different images and combining them all together. It takes time to Photoshop \dots I can just take those images separately and paint or draw them as draw them together, even though they're separate. [However] I think it would be easier to combine them but again, it's just too much work.''} - P5
\end{quote}
% P9, who earlier expressed discomfort using digital art tools, iteratively incorporated reference images into their pieces. These efforts to stitch together multiple static references underscored both the importance of having cohesive visual guides and the limits of existing tools, setting the stage for participants’ interest in more dynamic or interactive solutions.

Some used interactive references in place of multiple reference perspectives. For example, P1 explained that if they needed a reference \textit{``for human anatomy, I would personally go to download an app, like a medical app \dots because it's a 3d map. You can just rotate around the body \dots you can see all the muscles and how they work''} (P1). These examples illustrate how existing tools for combining or interacting with reference images often only supported fragments of participants' needs, leading them to use workarounds for fitting pieces together. While such strategies helped approximate visions, participants often required significant effort to interpret or manually blend references.

% opportunities participants suggest
While P1 was resourceful in finding such references, other artists expressed desires for a dedicated and accessible tool. Some envisioned interactive \textbf{systems for manipulating poses, perspectives, or stylistic features in real time}. When reflecting on an art piece they completed, P6 expressed that \textit{``it would have been really useful to have a pose of that thing falling in a realistic style''} that shows a \textit{``puppet reaching into the air with that particular pose from the angle that I want''} (P6). P8 shared a similar need to manipulate a human-like puppet and \textit{``move the face to make it mad''} (P8). 
These suggestions highlight artists' desire for tools that not only provide static references but also allow them to manipulate and combine references dynamically.


\subsubsection{Perspective and Feedback from Another Artist.}

Artists often struggled with identifying errors or areas for improvement on their own -- an artifact of lacking \textbf{timely, supportive feedback that matched their artistic intentions}. They highlighted unique challenges in this phase --- e.g., difficulty redoing parts of a piece without completely starting over or managing tools: \textit{``something that could have helped me catch that this arm had kind of [from a] wonky perspective before it got so far that I already colored and shaded -- it would have saved me a lot of time.''} (P3). This reflects a broader challenge of how artists access feedback and whose perspectives they can rely on.

% suggestions from participants
Several participants expressed a desire for feedback that is \textbf{context-aware, flexible, and intentional}. For many, feedback is most valuable when it comes from someone with technical expertise or an “artist’s eye”, who offer objective critique on aspects like balance, composition, or technique. P8 often sought input from her boyfriend, an art major with formal training, while P4 appreciated informal encouragement and more serious critique in group art settings. These artists found immense value in outside perspectives, especially when they had become too focused and needed fresh eyes to spot issues or opportunities. The desire for feedback also extended to compliments, encouragement, and casual observations from peers, especially in group settings or informal art gatherings. P4 explained that sometimes even disagreements with others were helpful: \textit{``you might not initially be in alignment with somebody. They might tell you something [about your art]. You're like, `okay, I could see that.' [You] try that and you [realize] `oh, actually, I love it''} (P4). Together, these accounts illustrate the importance of feedback on technical accuracy, motivation and perspective-taking, highlighting the multifaceted role of outside input in sustaining creative practice.


Participants imagined tools that could replicate the sensitivity of an ‘\textit{artist’s eye}’, offering feedback that was \textbf{flexible, contextualized, and tailored to their intentions}. For example, P11 wanted a tool that could help encourage and motivate them: \textit{``[When] I want to give up, the tool is like, `no, you can do it''} feeling that such a tool would \textit{``be so fun. It'd be so cute, too. I feel like people would like that''} (P11). \todo{can add more ideas for this from the validation sessions} Overall, in imagining these tools, participants emphasized the need for both constructive and uplifting feedback, surfacing opportunities for tools that provide the sharpness of an artist’s eye or support with reassurances.



\subsubsection{Support to Learn Art Techniques and Styles.}

% Participants also described struggles with 
Learning new techniques and styles on their own was a common among novices, with some finding it difficult to bridge the gap between observing others’ work and applying those methods in practice. This challenge was especially acute when artists attempted to adapt across mediums or translate from one stylistic tradition to another. For instance, P2 noted that \textit{``I'm not an art professional, so sometimes I would still struggle with techniques of painting certain things ... originally, I was learning traditional Chinese watercolor, and that's very different from Western watercolor''} (P2). To manage these difficulties, artists relied on practices such as copying interpretations of specific subjects when first learning (P2), recreating works through ``artist studies,'' following tutorials, or adapting pieces into different styles. Watching time-lapse or process videos (P11, P12) also helped demystify the effort behind finished pieces, making the process feel more approachable. Yet, translating observation into implementation was not always straightforward -- some expressed frustration when tutorials or advice were either too advanced or not grounded in the practical realities of their tools and materials (P2, P7). 

As such, participants emphasized the importance of resources that tailor to their current goals and level of experience, whether through step-by-step guidance, adaptive feedback, or examples that reveal works-in-progress rather than only finished pieces.
% To address this challenge, participants envisioned tools that could scaffold learning by breaking techniques into progressive steps, tailoring feedback to their level, and providing contextual guidance as they practiced. 
P7 described an ideal tool they would use:
\begin{quote}
    \textit{``ideally you have this tool that's watching over you while you paint, or while you are making art. It learns to provide you feedback at the right moment so that you're not being handheld the whole way, but, at the same time, you're not like getting lost. At the same time it figures out what [you're] good at, what you still need to spend some timeimproving on.''} - P7
\end{quote}
\todo{opportunity here to add more data from validation sessions}
Such opportunities highlight a desire for resources that are not only instructive but also responsive to an artist’s goals and intentions.


